\documentclass{article}

%% Language and font encodings
\usepackage[english]{babel}
\usepackage[utf8x]{inputenc}
\usepackage[T1]{fontenc}
\usepackage{graphicx}


%% Sets page size and margins
\usepackage[top=1in,bottom=1in,left=1in,right=1in,]{geometry}

%% Useful packages
\usepackage{amsmath}
\usepackage{graphicx}
\usepackage[colorinlistoftodos]{todonotes}
\usepackage[colorlinks=true, allcolors=blue]{hyperref}

\title{Project Proposal}
\author{Author}
\begin{document}

\maketitle

\section{Introduction}
What is the problem and why is it interesting and/or useful?

\section{Related Work}
What work has been done in the past to solve this problem? What is the consensus approach and how well does it perform? What datasets are used and what are their limitations?

\section{Objectives}
What do you want to achieve this semester? What do you want to achieve if you were to puruse this project over multiple semesters?

\section{Preliminary Analysis}
Get familiar with your data and establish some simple baselines:
\begin{itemize}
    \item What sort of performance could you achieve by randomly guessing? What about linear models?
    \item What are the distributions of some of the features and labels? Are any particularly skewed?
    \item Justify that the objectives are realistic
\end{itemize}

\begin{thebibliography}{9}

\bibitem{ResNet} Kaiming He, Xiangyu Zhang, Shaoqing Ren, and Jian Sun. Deep Residual Learning for Image Recognition. \href{https://arxiv.org/pdf/1808.03233.pdf}{https://arxiv.org/pdf/1808.03233.pdf}.

\end{thebibliography}

\end{document}